\chapter{Miniproject description}\label{ch:miniproject_description}
\begin{center}
    \bf
    Miniproject description \\ Sensors and Systems at ES8 \\Spring 2024
    \end{center}
    
    \begin{tabularx}{\linewidth}{lX}
    \bf Title & Determine speed and heading with GNSS \\
    \bf Participants & Jens Moeslund Larsen Gr. 823, Albert Werner Laursen Gr. 823  \\
    \bf Description & The project aims to develop a tracking module for high altitude balloon tests for use in future tests by the student satlab organization. The project aims to use a GNSS receiver and a barometer to estimate the position of the balloon. The position will be used for tracking the balloon, and for providing feedback for pointing directive antennas at the balloon. For improving the estimate of the speed and heading of the balloon, a predictive filter will be used. The project will be coded using the Rust programming language as this is the chosen language by the student satlab organization. The Rust programming language also includes multiple features for memory safe programming, heapless data structures and a strong type system. The system will eventually need to communicate with other systems on the balloon, and for this reason, the tasks will be designed with real time requirements in mind.\\
    \bf Target user & The intended target users for this product is the students in the student satlab organization. The users are therefore expected to have a decent amount of knowledge on the topic and the user interface can therefore be more focused on the speed instead of user friendliness. \\
    \bf IDE & Visual Studio Code \\
    \bf Sensors & Ms5611 Barometric pressure sensor and the NEO-6M GNSS module. \\
    \bf Lectures & Lecture 16, Lecture 15, Lecture 6 \\
    \end{tabularx}
    
    Note that the above information can be changed during the miniproject if needed. The purpose is for reviewing your decision, not to bind you to specific sensors and/or curriculum.