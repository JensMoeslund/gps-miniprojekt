\chapter{Filter}\label{ch:filter}
This chapter will give an overview of the kalman filter implemented in this project.


\section{System model}\label{sec:system-model}
The system model used in this project is a linear model.
The state-space representation of the system is given by the following equations:

\begin{align}
    \mathbf{x}_{k+1} &= \mathbf{A}\mathbf{x}_k + \mathbf{w}_k\\
    \mathbf{z}_k &= \mathbf{H}\mathbf{x}_k + \mathbf{v}_k
\end{align}

Where:
\begin{itemize}
    \item $\mathbf{x}_k$ is the state vector at time $k$.
    \item $\mathbf{A}$ is the state transition matrix.
    \item $\mathbf{w}_k$ is the process noise vector at time $k$.
    \item $\mathbf{z}_k$ is the observation vector at time $k$.
    \item $\mathbf{H}$ is the observation matrix.
    \item $\mathbf{v}_k$ is the observation noise vector at time $k$.
\end{itemize}

The above parameters will be explained in more detail in the following sections.


\section{State vector}\label{sec:state-vector}
The state vector $\mathbf{x}_k$ comprises the following kinematic variables:

\begin{align}
    \mathbf{x}_k =
    \begin{bmatrix}
       \phi_k          &
       \lambda_k       &
       r_k             &
       \dot{\phi}_k    &
       \dot{\lambda}_k &
       \dot{r}_k
    \end{bmatrix}^T
\end{align}

Where:
\begin{itemize}
    \item $\phi_k$ is the latitude at time $k$.
    \item $\lambda_k$ is the longitude at time $k$.
    \item $r_k$ is the altitude at time $k$.
    \item $\dot{\phi}_k$ is the velocity in the direction of latitude at time $k$.
    \item $\dot{\lambda}_k$ is the velocity in the direction of longitude at time $k$.
    \item $\dot{r}_k$ is the velocity in the direction of altitude at time $k$.
\end{itemize}

\section{State Transition Matrix}
The state transition matrix $\mathbf{A}$ is given by the kinematic equations of motion.
The kinematic equations of motion are given by the following equations:

\begin{align}
    \mathbf{x}_{k+1} =
    \begin{bmatrix}
        \phi_{k+1}          \\
        \lambda_{k+1}       \\
        r_{k+1}             \\
        \dot{\phi}_{k+1}    \\
        \dot{\lambda}_{k+1} \\
        \dot{r}_{k+1}
    \end{bmatrix}
    &=
    \begin{bmatrix}
        \phi_k + \dot{\phi}_k\Delta t \\
        \lambda_k + \dot{\lambda}_k\Delta t \\
        r_k + \dot{r}_k\Delta t \\
        \dot{\phi}_k \\
        \dot{\lambda}_k \\
        \dot{r}_k
    \end{bmatrix}
\end{align}

From the above equations, the state transition matrix $\mathbf{A}$ is derived as follows:

\begin{align}
    \mathbf{A} =
    \begin{bmatrix}
        1 & 0 & 0 & \Delta t & 0 & 0 \\
        0 & 1 & 0 & 0 & \Delta t & 0 \\
        0 & 0 & 1 & 0 & 0 & \Delta t \\
        0 & 0 & 0 & 1 & 0 & 0 \\
        0 & 0 & 0 & 0 & 1 & 0 \\
        0 & 0 & 0 & 0 & 0 & 1
    \end{bmatrix}
\end{align}


\section{Observation Vector}
The observation vector $\mathbf{z}_k$ comprises the following variables:

\begin{align}
    \mathbf{z}_k =
    \begin{bmatrix}
        \mathtt{lat\_g}_k &
        \mathtt{lon\_g}_k &
        \mathtt{alt\_g}_k &
        \mathtt{v\_alt\_g}_k &
        \mathtt{alt\_b}_k &
        \mathtt{v\_alt\_b}_k &
        \mathtt{v\_lat\_g}_k &
        \mathtt{v\_lon\_g}_k
    \end{bmatrix}^T
\end{align}

Where:
\begin{itemize}
    \item $\mathtt{lat\_g}_k$ is the latitude from the GNSS module at time $k$.
    \item $\mathtt{lon\_g}_k$ is the longitude from the GNSS module at time $k$.
    \item $\mathtt{alt\_g}_k$ is the altitude from the GNSS module at time $k$.
    \item $\mathtt{v\_alt\_g}_k$ is the vertical velocity from the GNSS module at time $k$.
    \item $\mathtt{alt\_b}_k$ is the altitude from the barometer at time $k$.
    \item $\mathtt{v\_alt\_b}_k$ is the vertical velocity from the barometer at time $k$.
    \item $\mathtt{v\_lat\_g}_k$ is the velocity in the direction of latitude from the GNSS module at time $k$.
    \item $\mathtt{v\_lon\_g}_k$ is the velocity in the direction of longitude from the GNSS module at time $k$.
\end{itemize}


\section{Observation Matrix}
The observation matrix $\mathbf{H}$ is derived from the observation vector $\mathbf{z}_k$.
Assuming a perfect observation, the observation vector $\mathbf{z}_k$ is given by the following equation:

\begin{align}
    \mathbf{z}_k =
    \begin{bmatrix}
        \mathtt{lat\_g}_k \\
        \mathtt{lon\_g}_k \\
        \mathtt{alt\_g}_k \\
        \mathtt{v\_alt\_g}_k \\
        \mathtt{alt\_b}_k \\
        \mathtt{v\_alt\_b}_k \\
        \mathtt{v\_lat\_g}_k \\
        \mathtt{v\_lon\_g}_k
    \end{bmatrix}
    =
    \begin{bmatrix}
        \phi_k \\
        \lambda_k \\
        r_k \\
        \dot{r}_k \\
        r_k \\
        \dot{r}_k \\
        \dot{\phi}_k \\
        \dot{\lambda}_k
    \end{bmatrix}
\end{align}

From the above equation, the observation matrix $\mathbf{H}$ is derived as follows:

\begin{align}
    \mathbf{H} =
    \begin{bmatrix}
        1 & 0 & 0 & 0 & 0 \\
        0 & 1 & 0 & 0 & 0 \\
        0 & 0 & 1 & 0 & 0 \\
        0 & 0 & 0 & 0 & 1 \\
        0 & 0 & 1 & 0 & 0 \\
        0 & 0 & 0 & 0 & 1 \\
        0 & 0 & 0 & 1 & 0
    \end{bmatrix}
\end{align}


\section{Noise Vectors}
The process noise vector $\mathbf{w}_k$ and the observation noise vector $\mathbf{v}_k$ are assumed to be wide-sense stationary white noise processes.
The process noise vector $\mathbf{w}_k$ is given by the covariance matrix $\mathbf{Q}$.
The observation noise vector $\mathbf{v}_k$ is given by the covariance matrix $\mathbf{R}$.

The covariance matrices $\mathbf{Q}$ and $\mathbf{R}$ will be estimated using the data collected from the GNSS module and the barometer

\section{Kalman Filter}
The Kalman filter is a recursive algorithm that can be split into two steps: the prediction step and the update step.

The update step is given by the following equations:

\begin{align}
    \mathbf{\tilde{z}}_k &= \mathbf{z}_k - \mathbf{H}\mathbf{\tilde{x}}_{k-1}\\
    \mathbf{S}_k &= \mathbf{H}\mathbf{P}_{k-1}\mathbf{H}^T + \mathbf{R}\\
    \mathbf{K}_k &= \mathbf{P}_{k-1}\mathbf{H}^T\mathbf{S}_k^{-1}\\
    \mathbf{\tilde{x}}_k &= \mathbf{\tilde{x}}_{k-1} + \mathbf{K}_k\mathbf{\tilde{z}}_k\\
    \mathbf{P}_k &= (\mathbf{I} - \mathbf{K}_k\mathbf{H})\mathbf{P}_{k-1}
\end{align}

Where:
\begin{itemize}
    \item $\mathbf{\tilde{z}}_k$ is the innovation vector at time $k$.
    \item $\mathbf{S}_k$ is the innovation covariance matrix at time $k$.
    \item $\mathbf{K}_k$ is the Kalman gain at time $k$.
    \item $\mathbf{P}_k$ is the error covariance matrix at time $k$.
\end{itemize}

The prediction step is given by the following equations:

\begin{align}
    \mathbf{\tilde{x}}_k &= \mathbf{A}\mathbf{\tilde{x}}_{k-1}\\
    \mathbf{P}_k &= \mathbf{A}\mathbf{P}_{k-1}\mathbf{A}^T + \mathbf{Q}
\end{align}

Where:
\begin{itemize}
    \item $\mathbf{\tilde{x}}_k$ is the predicted state vector at time $k$.
    \item $\mathbf{P}_k$ is the predicted error covariance matrix at time $k$.
\end{itemize}




