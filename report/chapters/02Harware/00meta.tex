\chapter{Hardware}\label{ch:hardware}
This chapter will give an overview of the hardware used for implementing a Kalman filter for the purpose of tracking a high altitude balloon.
It will also include answers to the SSEQ (Standard Sensor Exercise Questions) for the sensors used in the project.

\section{MCU}\label{sec:mcu}
The microcontroller used in this project is the STM32F446RE. The STM32F446RE is a 32-bit ARM Cortex-M4 microcontroller with a maximum CPU frequency of 180 MHz. 
The microcontroller has 512 KB of flash memory and 128 KB of RAM. 
The microcontroller has a variety of peripherals, including SPI, I2C, UART, and CAN. 
This particular microcontroller was chosen because it is the microcontroller used in the student satlab organization, where the project is intended to be used. 
%TODO: indsæt evt. billede af microcontrolleren

\section{Barometer}\label{sec:barometer}
The barometer used in this project is the MS5611. 
The MS5611 is a high-resolution pressure sensor. The purpose of the barometer is to measure the pressure, so an altitude can be calculated.
The MS5611 can measure pressure from 10 mbar to 1200 mbar, with a resolution of up to 0.012 mbar. %TODO: indsæt kilde
Since the expected maximum altitude of the balloon is around 15 km, the MS5611 is a suitable choice for this project.

\subsection{SSEQ for the MS5611}
\begin{enumerate}
    \item What is measured?
        \begin{enumerate}[(a)]
            \item Physical property/phenomenon measured?
            \item What are the units (physical phenomenon, output from sensor, desired signal)?
        \end{enumerate}

    \item Measuring
        \begin{enumerate}[(a)]
            \item What are the sampling frequency and resolution?
            \item Are measurements linear with the measured property?
        \end{enumerate}
            
    
\end{enumerate}






